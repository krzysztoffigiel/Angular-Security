
% Szkielet dla pracy inżynierskiej pisanej w języku polskim.

\documentclass[polish,bachelor,a4paper,oneside]{ppfcmthesis}


\usepackage[utf8]{inputenc}
\usepackage[OT4]{fontenc}


% Authors of the thesis here. Separate them with \and
\author{%
   Ignacy Iksiński \album{22222} \and 
   Wincent Woliński \album{22222} \and 
   Zdzisio Szmal \album{22222} \and 
   Barnaba Wojnowski \album{22222}}
\title{W zdrowym ciele zdrowy~duch}                   % Note how we protect the final title phrase from breaking
\ppsupervisor{prof.~dr hab.~inż.~Alojzy Wołodyjowski} % Your supervisor comes here.
\ppyear{2006}                                         % Year of final submission (not graduation!)


\begin{document}

% Front matter starts here
\frontmatter\pagestyle{empty}%
\maketitle\cleardoublepage%

% Blank info page for "karta dyplomowa"
\thispagestyle{empty}\vspace*{\fill}%
\begin{center}Tutaj przychodzi karta pracy dyplomowej;\\oryginał wstawiamy do wersji dla archiwum PP, w pozostałych kopiach wstawiamy ksero.\end{center}%
\vfill\cleardoublepage%

% Table of contents.
\pagenumbering{Roman}\pagestyle{ppfcmthesis}%
\tableofcontents* \cleardoublepage%

% Main content of your thesis starts here.
\mainmatter%

\chapter{Wstęp}

%Wstęp\footnote{Treść przykładowych rozdziałów została skopiowana
%z ,,zasad'' redakcji prac dyplomowych FCMu~\cite{fcm-red}.} do pracy powinien zawierać następujące elementy:
\section{Wprowadzenie}
Wybór tematu pracy magisterskiej spowodowany był chęcią udoskonalenia i poszerzenia swojej wiedzy na temat tworzenia nowoczesnych aplikacji WWW opartych na języku \textit{JavaScript} z wykorzystaniem technologii implementacji warstwy prezentacji oraz mechanizmów bezpieczeństwa z nimi powiązanych. Aktualny rynek pracy i ciagły rozwój technologii informatycznych (zwłaszcza tych internetowych) powodują stałe udoskonalanie aktualnych rozwiązań technologicznych, a co za tym idzie uodparnianie ich na problemy związane z bezpiecznym przechowywaniem danych. Kolejnym argumentem, który wskazuje na mój wybór jest fakt, że aktualnie pracuję na stanowisku web dewelopera, tak więc tworzenie bezpiecznych aplikacji internetowych jest zarówno źródłem mojego dochodu, jak i zainteresowań. W dzisiejszych czasach programista ma dostęp do wielu użytecznych bibliotek i rozwiązań, które w znacznym stopniu ułatwiają implementację najważniejszych mechanizmów komunikacji poszczególnych warstw aplikacji z serwerem, dlatego też postanowiłem omówić najważniejsze z nich wraz z krótkimi implementacjami podstawowych funkcjonalności stron WWW. 

%\begin{itemize}
%    \item krótkie uzasadnienie podjęcia tematu; 
 
%\end{itemize}

\noindent

\section{Cel projektu}
Za cel projektu postawiono analizę działania i bezpieczeństwa aplikacji opartych na języku \textit{JavaScript} w oparciu o proste implementacje podstawowych funkcjonalności serwisów WWW takich jak uwierzytelnianie, formularze użytkownika czy wymiana danych między warstwami prezentacji i aplikacji. Skupiono się na zrealizowaniu podstawowych mechanizmów prewencji przeciwko atakom z poziomu warstwy aplikacji. Zaproponowano również autorskie pomysły metod dodatkowego zabezpieczania stron WWW przeciwko nieporządanemu działaniu osób trzecich. Zestawiono sposoby realizacji powyższych rozwiązań w poszczególnych technologiach frontendowych i określono, która z nich jest potencjalnie najlepsza. Ostatecznie, zaproponowano ewentualne metody ulepszeń omawianych mechanizmów i podsumowano całokształt pracy.

\section{Struktura pracy}

Praca składa się z siedmiu rozdziałów. Pierwszy z nich zawiera wprowadzenie, przedstawia cel projektu oraz ogólną strukturę pracy. W drugim opisane są technologie wykorzystane do stworzenia testowej aplikacji i analizy wybranych zagadnień. Trzeci rozdział opisuje środowisko backendowe \textit{Node.js}. Przedstawia on sposób implementacji i uruchomienia serwera z wykorzystaniem bezpiecznego połączenia - HTTPS.  Kolejny, czwarty, opisuje mechanizmy bezpieczeństwa aplikacji internetowych opartych na języku \textit{JavaScript}. W rozdziale piątym przeanalizowano i zestawiono ze sobą wybrane technologie implementacji warstwy prezentacji. Rozdział szósty zawiera analizę hybrydowego frameworka \textit{Ionic} i opisuje wybrane mechanizmy bezpieczeństwa, oferowane przez funkcje natywne biblioteki. Ostatni, siódmy rozdział jest podsumowaniem całej pracy. Na końcu znajduje się spis literatury. Do pracy dołączono załącznik w~ postaci płyty DVD. 




\chapter{Technologie}

\section{Wprowadzenie}
W poniższym rozdziale opisano najważniejsze technologie opisane w pracy. W kolejnych sekcjach przedstawiono zestaw wymagań funkcjonalnych z~podziałem na biorących udział aktorów oraz~wymagania niefunkcjonalne.

\section{Technologie}

\subsection{Framework Angular 6}
Otwarty framework stworzony przez firmę \textit{Google}, wykorzystywany do tworzenia aplikacji SPA (\textit{Single Page Application}) zarówno na platformy internetowe, jak i na natywne aplikacje mobilne i desktopowe. \textit{Angular} oparty jest na języku \textit{JavaScript} przez co zyskał on wielką popularność wśród deweloperów, w dużej mierze ze względu na swoją prostotę. Jego struktura wymusza u programistów stosowanie dobrych praktyk pisania kodu i pomaga usestymatyzować na pozór skomplikowaną architekturę aplikacji webowych. Framework ten został napisany całkowicie w mocno typowanym języku \textit{TypeScript}, który jest transpilowany do wynikowego kodu \textit{JavaScript}. \textit{Angular} swoją strukturą zachęca do budowania maksymalnie odseparowanych od siebie części kodu i komponentów. Kolejną zaletą tego frameworku jest wieloplatformowość. Pozwala on bowiem na tworzenie stron internetowych, aplikacji webowych, aplikacji PWA (\textit{Progressive Web Application}), aplikacji mobilnych z użyciem bibliotek takich jak \textit{Cordova} czy też aplikacji desktopowych mogących odnosić się do funkcji systemu i lokalnych urządzeń. Wersja szósta frameworka \textit{Angular} wprowadza zmiany m.in. w rejestrowaniu serwisów, stosowaniu \textit{RxJS}, walidacji formularzy czy w komendach CLI (\textit{Command Line Interface}).
 
\subsection{TypeScript 3}
\textit{TypeScript} jest językiem kompilowanym do języka \textit{JavaScript}, który urchamiany jest w dowolnej przeglądarce, na serwerze \textit{Node.js} czy w innym silniku, który wspiera \textit{ECMAScript} w wersji trzeciej lub nowszej. Jest to język silnie i statycznie typowany, który pomaga deweloperom stosować dobre, programistyczne praktyki. Statyczne typowanie rozumiane jest przez to, że zmienne w tym języku mają nadane typy, które nie mogą ulec zmianie. Silne typowanie wprowadza nie tylko przejrzystość kodu i jego lepsze debugowanie, ale również takie możliwości jak \textit{IntelliSense}, czyli podpowiedzi. Zastosowanie \textit{TypeScript} pozwala pozbyć się przypadkowych błędów związanych z niepewnością odnośnie typów danych oraz eliminuje problemy związane ze skalowalnością. Umożliwia stosowanie interfejsów, które pozwalają podnieść poziom abstrakcji i uzyskać luźniejsze powiązania pomiędzy klasami w aplikacjach.

\subsection{HTML5}
HTML5 (\textit{HyperText Markup Language}) jest najnowszą wersją popularnego standardu opisującego język HTML. W stosunku do poprzedników zawiera on nowe elementy, atrybuty i zachowania. Pozwala na bardziej różnorodne tworzenie stron i aplikacji internetowych. Umożliwia nowoczesną komunikację z serwerem, pozwala stronom internetowym na bardziej efektywne przechowywanie danych lokalnie i w trybie offline czy zapewnie większą prędkość i lepszą optymalizację w wykorzystywaniu sprzętu komputerowego. HTML5 wprowadza nowe elementy sekcji takie jak \texttt{<section>}, \texttt{<article>}, \texttt{<nav>}, \texttt{<header>} czy \texttt{<footer>}. Ulepszone zostały m.in. formularze, wymuszenie poprawności API (\textit{Application Programming Interface}) czy znaczniki \texttt{<input>} i \texttt{<output>}, które zyskały nowe atrybuty, takie jak \texttt{email} oraz \texttt{password}. HTML5 zapewnia również uproszczone odtwarzanie plików audio i wideo. Oferuje dużo prostsze tworzenie i wyświetlanie grafiki przy użyciu znaczników \texttt{<canvas>}. Ważną nowością są też tzw. \textit{Web-Workers}, które umożliwiają wielowątkową obsługę przepływu danych.

\subsection{CSS3}
CSS3 (\textit{Cascading Style Sheets}) jest to kolejna wersja kaskadowych arkuszy stylów, wprowadzająca szereg udogodnień i rozszerzająca możliwości interakcji użytkownika ze stroną internetową. W porównaniu do poprzednich wersji, zmiany w CSS3 obejmują między innymi zastosowanie animowanych elementów czy wszelkiego rodzaju efektów graficznych, takich jak gradienty oraz cienie. 
CSS3 jest kompatybilny ze wszystkimi stylami objętymi standardem CSS2. Nie ma zatem potrzeby modyfikowania starszych stron przy przejściu na wyższą wersję kaskadowych arkuszy stylów. Trzecia wersja CSS zyskała również modułową budowę – specyfikacja zostaje podzielona na wiele różnych dokumentów, dzięki czemu rozwój odrębnych modułów aplikacji odbywa się odrębnie, a kod jest czysty i uporządkowany. 
Style CSS dodawane są do elementów na podstawie ich pozycji w drzewie dokumentu (\textit{Document Tree}). CSS jest w pełni kompatybilny z językiem HTML, co oznacza, że HTML strukturyzuje treść strony, natomiast CSS formatuje ją w odpowiedni sposób. CSS3 pozwala na całkowitą kontrolę układu graficznego dokumentów z poziomu tylko jednego arkusza stylów. Ponadto, programista ma bardziej precyzyjną kontrolę nad całym układem graficznym \cite{Css}.

\subsection{Bootstrap 4}
Jedna z najbardziej popularnych bibliotek dla języka HTML, CSS i \textit{JavaScript}. \textit{Bootstrap} to zestaw przydatnych narzędzi wykorzystywany do tworzenia responsywnych interfejsów aplikacji internetowych oraz mobilnych. Bazuje w dużej mierze na gotowych implementacjach HTML i CSS, które kompilowane są bezpośrednio z pliku \textit{Less}. Biblioteka może być wykorzystywana w celu stylizowania formularzy, tekstów, przycisków, elementów menu i wielu innych przydatnych komponentów stron internetowych. Jedną z najważniejszych cech charakteryzujących \textit{Bootstrap} jest to, że wprowadza on system siatek (\textit{Grid}) mający na celu usystematyzowanie położenia elementów na stronie. Wszystko opiera się na dzieleniu strony na rzędzy (\textit{rows}), a rzędy na kolumny (\textit{columns}). Szerokość każdej z kolumn określana jest liczbą, a suma wszystkich w rzędzie powinna wynosić 12. Framework korzysta z języka \textit{JavaScript}. 

\subsection{NodeJS 8}
\textit{NodeJS} jest środowiskiem uruchomieniowym stosowanym do tworzenia wysoce skalowalnych aplikacji internetowych, a w szczególności serwerów WWW napisanych w języku \textit{JavaScript}, dzięki czemu zyskał on sporą popularność. Wykorzystuje on asynchroniczny system wejścia-wyjścia i składa się z silnika stworzonego przez firmę \textit{Google}. \textit{NodeJS} umożliwia sprawne zarządzanie bibliotekami i ich zależnościami poprzez dostarczane oprogramowanie NPM (\textit{Node Package Manager}). Dodatkowo dzięki niemu możemy uruchamiać aplikacje napisane w \textit{JavaScript} na urządzeniu IoT (\textit{Internet of Things}) co jeszcze bardziej potęguje jego skalowalność i powszechność. Obecnie środowisko to używane jest przez światowych gigantów takich jak \textit{Google}, \textit{Microsoft}, \textit{Amazon} czy \textit{Netflix}. Ma swoje podstawy w języku C++ co daje możliwość zapewnienia odpowiedniego poziomu stabilności oraz szybkości pisanych aplikacji. 

\subsection{Express.js}
\textit{Express.js} jest frameworkiem typu \textit{open source} przeznaczonym do współpracy z \textit{Node.js}. Jest zaprojektowany do tworzenia aplikacji internetowych oraz przeznaczonych dla nich API (\textit{Application Programming Interface}). Jest standardem typu \textit{de facto} zastosowań serwerowych bazujących na \textit{Node.js}. \textit{Express.js} pozwala definiować tabele routingu w celu wykonywania różnego typu akcji w oparciu o metodę HTTP. Umożliwia ponadto dynamiczne renderowanie stron napisanych w języku HTML w oparciu o przekazywanie argumentów do szablonów. 

\subsection{Karma}
Narzędzie zbudowane w oparciu o serwer \textit{NodeJS} oraz technologię \textit{Socket.io}, służące do automatycznego uruchamiania testów tworzonych w języku \textit{JavaScript} w emulowanym środowisku przeglądarek internetowych. Za pomocą \textit{Karma} możemy uruchamiać testy na różnych środowiskach programistycznych - deweloperskim, produkcyjnym czy testowym. Przy użyciu narzędzia \textit{Istanbul} możliwy jest dostęp do informacji na temat pokrycia implementowanego kodu testami. \textit{Karma} jest pełnoprawnym środowiskiem testowym ułatwiającym szybkie i bezproblemowe testowanie kodu \textit{JavaScript}.

\subsection{Jasmine}
Framework typu \textit{behavior-driven development framework} dający programistom wiele przydatnych funkcji potrzebnych do testowania oprogramowania. Jest zintegrowany ze środowiskiem \textit{Karma} i pozwala na pisanie testów oprogramowania w sposób opisowy. Nie jest on zależny od środowiska \textit{JavaScript} i nie wymaga drzewa DOM (\textit{Document Object Model}). Jasmine pozwala nie tylko na pisanie testów jednostkowych, ale także testów typu \textit{e2e} (\textit{end-to-end}). 

\subsection{Apache Cordova}
\textit{Cordova} to w skrócie API (\textit{Application Programming Interface}) umożliwiające stworzenie natywnej aplikacji używając wyłącznie HTML, CSS oraz kodu \textit{JavaScript} przy jednoczesnym dostępie do komponentów urządzeń mobilnych, takich jak aparat, usługi geolokalizacji czy nawet książki kontaktów. Aplikacje stworzone w ten sposób mogą znaleźć się na urządzeniach mobilnych producentów najbardziej wiodących firm - \textit{iOS}, \textit{Android}, \textit{Windows Phone} czy \textit{BlackBerry}.

\subsection{LaTeX}
Oprogramowanie służące do tworzenia przejrzystych dokumentów tekstowych takich jak na przykład książki czy artykuły. Docelowym plikiem wyjściowym jest najczęściej plik w formacie PDF (\textit{Portable Document Format}). Cechą charakterystyczną jest tutaj fakt, że \textit{LaTeX} ma swój własny język programowania, za pomocą którego tworzone są dokumenty. Do wygenerowania dokumentów przydatne są narzędzia przetwarzające pliki źródłowe i generujące dokumenty wyjściowe. \textit{LaTeX} oferuje dostęp do szeregu pakietów umożliwiających znacznie prostsze i szybsze implementowanie bardziej złożonych elementów plików wyjściowych. Filozofia \textit{LaTeXa} zakłada, aby skupiać się nie na tym jak dokument ma wyglądać, a co ma zawierać. Do użytkownika należy tylko wprowadzenie struktury i zawartości dokumentu. 

\subsection{PBKDF2}
PBKDF2 (\textit{Password-Based Key Derivation Function 2}) jest popularnym algorytmem podobnym do algorytmu \textit{BCrypt}, zapewniającym porównywalny stopień bezpieczeństwa. PBKDF2 jest bezpieczną funkcją skrótu stosowaną w celach zabezpieczania bezprzewodowych sieci WiFi (WPA, WPA2). Algorytm ten w celu wygenerowania klucza pochodnego wykorzystuje pseudolosową funkcję, taką jak HMAC (\textit{Hash Message Authentication Code}) powtarzając operacje hashowania przez określoną, dużą liczbę iteracji. Algorytm PBKDF2 jest trudny do złamania za pomocą CPU (\textit{Central Processing Unit}), ale nie wymaga zbyt dużych zasobów pamięciowych i łatwo jest go zrównoleglić za pomocą GPU (\textit{Graphics Processing Unit}). Mimo to, że jest on wciąż stosowany to nie zaleca się stosowania go do nowych projektów. 

\subsection{Argon2}
Algorytm \textit{Argon2} został zwycięzcą \textit{Password Hashing Competition} i jako następca \textit{BCrypt} oraz \textit{Scrypt} jest obecnie zalecany do zabezpieczania haseł. \textit{Argon2} zawiera szereg zabezpieczeń przeciwko atakom typu \textit{Brute Force}. W przeciwieństwie do \textit{PBKDF2} algorytm ten wprowadza silną odporność nie tylko na ataki przy użyciu CPU, ale także GPU (wykorzystuje się konkretną odmianę \textit{Argon2d}). Samo użycie algorytmu jest niezwykle proste. WIele języków programowania oferuje dedykowane bilbioteki, które znacząco ułatwiają używanie \textit{Argon2} w implementacjach. 

\section{Narzędzia}
\subsection{Visual Studio Code 1.28}
\textit{Visual Studio Code} jest wieloplatformowym, prostym w obsłudze IDE (\textit{Integrated Development Environment}) stworzonym przez firmę \textit{Microsoft}. Łączy on w sobie prostotę edytora kodu źródłowego z potężnym środowiskiem deweloperskim oferującym mechanizm \textit{IntelliSense} czy mechanizm debugowania kodu. \textit{Visual Studio Code} oferuje szereg skrótów klawiszowych i snippetów ułatwiających szybsze i bardziej intuicyjne implementowanie oprogramowania. Autorzy udostępnili mnóstwo udogodnień wspierających pracę w zespole, takich jak integracja z systemem kontroli wersji \textit{Git} czy też rozproszone współdzielenie kodu. Oprogramowanie połączone jest bezpośrednio z uaktualnianym na bieżąco repozytorium paczek i pakietów ułatwiających programowanie w danym języku i technologii. Dodatkowo, \textit{VSCode} oferuje przyjazne GUI (\textit{Graphical User Interface}), z przejrzystym eksploratorem drzewa plików i mapą zawartości pliku, ułatwiającą szybsze wykrycie błędów i ostrzeżeń w kodzie \cite{Vsc}.

\subsection{github.com}
Serwis internetowy stworzony dla projektów programistycznych, który wykorzystuje system kontroli wersji \textit{Git}. Jego implementacja ma podłoże w języku \textit{Erlang} z wykorzystaniem frameworka \textit{Ruby on Rails}. Github umożliwia darmowy hosting plików oraz płatne, prywatne repozytoria. Platforma oferuje szereg statystyk powiązanych z implementowanym kodem źródłowym, mechanizm typu \textit{bugtracker}, możlwiość pobierania repozytoriów, rozgałęziania (dzielenia) pracy pomiędzy członków zespołu programistycznego czy późniejszego ich łączenia. Dodatkowo serwis ten oferuje usługę zwaną \textit{Github Pages} służącą do szybkiego tworzenia stron internetowych kompilowanych na podstawie kodu zawartego w repozytorium. Dzięki \textit{github.com} mamy możliwość bezpośredniej kontroli nad tworzonym kodem oprogramowania i dostęp do historii pracy co znacząco ułatwia zarządzanie projektem programistycznym. 

\subsection{Auth0}
Serwis \textit{Auth0} umożliwia połączenie z dowolną aplikacją w celu zaoferowania usług uwierzytelniania użytkowników. Oferuje on metody logowania i rejestracji w serwisie za pomocą tradycyjnego loginu (adresu e-mail) i hasła, bądź mediów społecznościowych takich jak \textit{Google}, \textit{Facebook} czy \textit{Twitter}. Domyślny protokół używany do integracji serwisu \textit{Auth0} z aplikacją użytkownika i~ pózniejszego uwierzytelniania to OIDC (\textit{OpenID Connect}). Używa on prostych \textit{tokenów} identyfikacyjnych w formacie JSON (\textit{JavaScript Object Notation}). Wymiana danych odbywa się przy użyciu JWT (\textit{JSON Web Token}), który zawiera wszystkie dane identyfikacyjne użytkownika \cite{Auth}.

\subsection{TexStudio}
\textit{TexStudio} jest zintegrowanym środowiskiem służącym do tworzenia dokumentów w języku \textit{LaTeX}. Program posiada szereg funkcji mających na celu ułatwienie tworzenia tekstów, a w ich skład wchodzą między innymi podświetlanie składni, zintegrowana przeglądarka, system sprawdzania referencji czy zintegrowane, zewnętrzne repozytorium pakietów i rozszerzeń języka \textit{LaTeX}.
\input{03-praca-wlasna.tex}
\input{04-zakonczenie.tex}

% All appendices and extra material, if you have any.
\cleardoublepage\appendix%
\chapter{Załączniki}





% Bibliography (books, articles) starts here.
\bibliographystyle{plalpha}{\raggedright\sloppy\small\bibliography{bibliography}}

% Colophon is a place where you should let others know about copyrights etc.
\ppcolophon

\end{document}
