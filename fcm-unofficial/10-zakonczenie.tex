\chapter{Zakończenie}

Ustosunkowując się do założeń podanych we wstępie pracy, dołożono wszelkich starań, aby oferowane przez aplikacje funkcjonalności działały zgodnie z ich przeznaczeniem oraz zaprezentowane były za pomocą prostego i przejrzystego interfejsu. Aplikacja ma być jedynie jednym z~ czynników przyczyniających się do postępów w leczeniu cukrzycy. 

Poprzez oferowanie czytelnych wykresów podsumowujących wszystkie, dotychczasowe pomiary wprowadzone przez użytkownika oraz tabeli zestawień grupującej wartości pomiarów osoba korzystająca z aplikacji wyciągać ma poglądowe wnioski na temat tego jak przebiega jej choroba od momentu założenia konta w systemie. Ponadto, dzięki dostępnym kalkulatorom BMI (\textit{Body Mass Index}), wymienników węglowodanowych czy wymienników białkowo-tłuszczowych opartych na stale rozwijanej przez użytkowników bazie produktów i danych wprowadzanych przez korzystającą z nich osobę możliwe jest określenie jaki stopień współczynnika masy ciała posiada dany użytkownik czy też jakie produkty mogą być spożyte w danej chwili tak, aby nie zakłócić porządku dziennej diety i nie pogłębić stopnia zaawansowania cukrzycy. 

Jeżeli chodzi o dalsze etapy rozwoju aplikacji -- autor chciałby rozszerzyć jej funkcjonalność o~ moduł oparty na zdobywaniu punktów przez użytkowników w zamian za postępy w leczeniu cukrzycy. Funkcjonalność ta mogłaby mieć znaczący wpływ na motywację i~ sukcesywne korzystanie z aplikacji w zamian za systematyczne prowadzenie pomiarów. Dodatkowo, jest w~ planach dodanie modułu skanowania kodów kreskowych produktów przez użytkowników za pomocą kamerki internetowej, bądź kamery telefonu i~ automatycznego dodawania ich do bazy danych systemu. Ułatwiłoby to znacznąco proces wprowadzania nowego produktu.
