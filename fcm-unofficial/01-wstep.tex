
\chapter{Wstęp}

%Wstęp\footnote{Treść przykładowych rozdziałów została skopiowana
%z ,,zasad'' redakcji prac dyplomowych FCMu~\cite{fcm-red}.} do pracy powinien zawierać następujące elementy:
\section{Wprowadzenie}
Wybór tematu pracy magisterskiej spowodowany był chęcią udoskonalenia i poszerzenia swojej wiedzy na temat tworzenia nowoczesnych aplikacji WWW opartych na języku \textit{JavaScript} z wykorzystaniem technologii implementacji warstwy prezentacji oraz mechanizmów bezpieczeństwa z nimi powiązanych. Aktualny rynek pracy i ciagły rozwój technologii informatycznych (zwłaszcza tych internetowych) powodują stałe udoskonalanie aktualnych rozwiązań technologicznych, a co za tym idzie uodparnianie ich na problemy związane z bezpiecznym przechowywaniem danych. Kolejnym argumentem, który wskazuje na mój wybór jest fakt, że aktualnie pracuję na stanowisku web dewelopera, tak więc tworzenie bezpiecznych aplikacji internetowych jest zarówno źródłem mojego dochodu, jak i zainteresowań. W dzisiejszych czasach programista ma dostęp do wielu użytecznych bibliotek i rozwiązań, które w znacznym stopniu ułatwiają implementację najważniejszych mechanizmów komunikacji poszczególnych warstw aplikacji z serwerem, dlatego też postanowiłem omówić najważniejsze z nich wraz z krótkimi implementacjami podstawowych funkcjonalności stron WWW. 

%\begin{itemize}
%    \item krótkie uzasadnienie podjęcia tematu; 
 
%\end{itemize}

\noindent

\section{Cel projektu}
Za cel projektu postawiono analizę działania i bezpieczeństwa aplikacji opartych na języku \textit{JavaScript} w oparciu o proste implementacje podstawowych funkcjonalności serwisów WWW takich jak uwierzytelnianie, formularze użytkownika czy wymiana danych między warstwami prezentacji i aplikacji. Skupiono się na zrealizowaniu podstawowych mechanizmów prewencji przeciwko atakom z poziomu warstwy aplikacji. Zaproponowano również autorskie pomysły metod dodatkowego zabezpieczania stron WWW przeciwko nieporządanemu działaniu osób trzecich. Zestawiono sposoby realizacji powyższych rozwiązań w poszczególnych technologiach frontendowych i określono, która z nich jest potencjalnie najlepsza. Ostatecznie, zaproponowano ewentualne metody ulepszeń omawianych mechanizmów i podsumowano całokształt pracy.

\noindent
\section{Różnice pomiędzy istniejącymi aplikacjami}

Główną zaletą projektu jest jego dostępność. Większość aplikacji dostępnych na rynku ograniczona jest do jednego systemu operacyjnego. Tutaj celem było stworzenie aplikacji dostępnej z~ poziomu przeglądarki internetowej, co znacznie poszerza jej zakres kompatybilności. 


\section{Struktura pracy (ToDo)}

Praca składa się z X rozdziałów. Pierwszy zawiera wprowadzenie, ukazuje cel projektu oraz przedstawia różnice pomiędzy istniejącymi aplikacjami. W drugim opisane są technologie wykorzystane do stworzenia aplikacji, wymagania funkcjonalne z podziałem na aktorów i niefunkcjonalne wraz z bardziej szczegółową analizą podobnych aplikacji istniejących na rynku. Trzeci rozdział opisuje procesy zachodzące w systemie. Kolejny, czwarty, przedstawia dokładny opis bazy danych. W piątym rozdziale omówiono panel administratora aplikacji. W następnych dwóch przedstawiono kolejno panel użytkownika zarejestrowanego oraz panel użytkownika niezarejestrowanego i niezalogowanego. Rozdział ósmy dotyczy bezpieczeństwa aplikacji i opisuje wszystkie aspekty z nim związane. W~ rozdziale dziewiątym przedstawiono testy funkcjonalne wykonane na aplikacji. Ostatni, dziesiąty rozdział jest podsumowaniem całej pracy. Omówiono tam także dalsze etapy rozwoju aplikacji. Na końcu znajduje się spis literatury. Do pracy dołączono załącznik w~ postaci płyty DVD. 


