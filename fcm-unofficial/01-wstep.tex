
\chapter{Wstęp}

%Wstęp\footnote{Treść przykładowych rozdziałów została skopiowana
%z ,,zasad'' redakcji prac dyplomowych FCMu~\cite{fcm-red}.} do pracy powinien zawierać następujące elementy:
\section{Wprowadzenie}
Wybór tematu pracy magisterskiej spowodowany był chęcią udoskonalenia i poszerzenia swojej wiedzy na temat tworzenia nowoczesnych aplikacji WWW opartych na języku \textit{JavaScript} z wykorzystaniem technologii implementacji warstwy prezentacji oraz mechanizmów bezpieczeństwa z nimi powiązanych. Aktualny rynek pracy i ciagły rozwój technologii informatycznych (zwłaszcza tych internetowych) powodują stałe udoskonalanie aktualnych rozwiązań technologicznych, a co za tym idzie uodparnianie ich na problemy związane z bezpiecznym przechowywaniem danych. Kolejnym argumentem, który wskazuje na mój wybór jest fakt, że aktualnie pracuję na stanowisku web dewelopera, tak więc tworzenie bezpiecznych aplikacji internetowych jest zarówno źródłem mojego dochodu, jak i zainteresowań. W dzisiejszych czasach programista ma dostęp do wielu użytecznych bibliotek i rozwiązań, które w znacznym stopniu ułatwiają implementację najważniejszych mechanizmów komunikacji poszczególnych warstw aplikacji z serwerem, dlatego też postanowiłem omówić najważniejsze z nich wraz z krótkimi implementacjami podstawowych funkcjonalności stron WWW. 

%\begin{itemize}
%    \item krótkie uzasadnienie podjęcia tematu; 
 
%\end{itemize}

\noindent

\section{Cel projektu}
Za cel projektu postawiono analizę działania i bezpieczeństwa aplikacji opartych na języku \textit{JavaScript} w oparciu o proste implementacje podstawowych funkcjonalności serwisów WWW takich jak uwierzytelnianie, formularze użytkownika czy wymiana danych między warstwami prezentacji i aplikacji. Skupiono się na zrealizowaniu podstawowych mechanizmów prewencji przeciwko atakom z poziomu warstwy aplikacji. Zaproponowano również autorskie pomysły metod dodatkowego zabezpieczania stron WWW przeciwko nieporządanemu działaniu osób trzecich. Zestawiono sposoby realizacji powyższych rozwiązań w poszczególnych technologiach frontendowych i określono, która z nich jest potencjalnie najlepsza. Ostatecznie, zaproponowano ewentualne metody ulepszeń omawianych mechanizmów i podsumowano całokształt pracy.

\section{Struktura pracy}

Praca składa się z siedmiu rozdziałów. Pierwszy z nich zawiera wprowadzenie, przedstawia cel projektu oraz ogólną strukturę pracy. W drugim opisane są technologie wykorzystane do stworzenia testowej aplikacji i analizy wybranych zagadnień. Trzeci rozdział opisuje środowisko backendowe \textit{Node.js}. Przedstawia on sposób implementacji i uruchomienia serwera z wykorzystaniem bezpiecznego połączenia - HTTPS.  Kolejny, czwarty, opisuje mechanizmy bezpieczeństwa aplikacji internetowych opartych na języku \textit{JavaScript}. W rozdziale piątym przeanalizowano i zestawiono ze sobą wybrane technologie implementacji warstwy prezentacji. Rozdział szósty zawiera analizę hybrydowego frameworka \textit{Ionic} i opisuje wybrane mechanizmy bezpieczeństwa, oferowane przez funkcje natywne biblioteki. Ostatni, siódmy rozdział jest podsumowaniem całej pracy. Na końcu znajduje się spis literatury. Do pracy dołączono załącznik w~ postaci płyty DVD. 


