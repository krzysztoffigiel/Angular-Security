
\chapter{Technologie}

\section{Wprowadzenie}
W poniższym rozdziale opisano najważniejsze technologie opisane w pracy. W kolejnych sekcjach przedstawiono zestaw wymagań funkcjonalnych z~podziałem na biorących udział aktorów oraz~wymagania niefunkcjonalne.

\section{Technologie}

\subsection{Framework Angular}
\textit{Angular} jest to frontendowy zestaw bibliotek, który służy do implementacji stron typu SPA (\textit{Single Page Application}). Cały DOM (\textit{Document Object Model}) strony ładowany jest wówczas jednorazowo w trakcie jej uruchomienia. Dzięki takiemu rozwiązaniu można w wygodny sposób korzystać ze strony internetowej bez jej ciągłego przeładowywania. \textit{Angular} oferuje bardzo proste mechanizmy manipulacji DOMem. Dzięki instrukcjom sterującym, takim jak \textit{ngIf}, \textit{ngFor} itp. możemy w dowolny sposób sterować treścią wyświetlaną na stronie. Kolejnym udogodnieniem oferowanym przez framework \textit{Angular} jest mechanizm \textit{Data binding}, dzięki któremu dowolna zmiana zmiennej z poziomu szablonu powoduje automatyczne odświeżanie jej wartości w przeglądarce internetowej. Cała struktura \textit{Angulara} wymusza u programistów stosowanie dobrych praktyk pisania kodu i pomaga usystematyzować na pozór skomplikowaną architekturę projektu. Kod pisany jest w języku \textit{Typescript}, który kompilowany (transpilowany) jest do wynikowego kodu \textit{JavaScript}. Pomimo tego, że mechanizm kompilacji spowalnia nieco działanie aplikacji, to dostarcza on szereg udogodnień oferowanych przez \textit{ECMAScript}, takich jak na przykład typowanie. Typowanie pozwala w bardzo prosty sposób zapanować nad kodem i znacznie sprawniej lokalizować ewentualne błędy. Kolejnym z udogodnień jest biblioteka \textit{RxJS} (\textit{Reactvie Extensions for JavaScript}), która oferuje zestawy narzędzi do programowania asynchronicznego czy opartego na zdarzeniach. Ogromną zaletą \textit{Angulara} jest jego wieloplatformowość - pozwala on bowiem na tworzenie aplikacji webowych, aplikacji PWA (\textit{Progressive Web App}), aplikacji mobilnych, korzystających z funkcji natywnych urządzeń (framework \textit{Cordova}) czy też aplikacji desktopowych z wykorzystaniem frameworka \textit{Electron}. \cite{Angular}
 
\subsection{TypeScript}
\textit{TypeScript} to zorientowany obiektowo, oparty na klasach nadzbiór języka \textit{Javascript} stworzony przez firmę \textit{Microsoft}. Oznacza to, że kod pisany w \textit{TypeScripcie} kompilowany (transpilowany) jest do wynikowego kodu \textit{JavaScript}, którego wersję ustalić może sam użytkownik. Kod pisany w tym języku jest znacznie przejrzysty i łatwiejszy do przeanalizowania. Współpracuje on z różnymi mechanizmami podpowiedzi, takimi jak \textit{IntelliSense}, przez co programowanie staje się znacznie łatwiejsze. \textit{Typescript} obsługuje pliki nagłówkowe, dodające informacje o typie do istniejących już bibliotek \textit{JavaScript}. Dzięki takiemu rozwiązaniu takie biblioteki jak \textit{jQuery} czy \textit{Node.js} można używać bez żadnych przeszkód. Język ten oferuje również dodatkowe mechanizmy programistyczne, takie jak klasy, interfejsy, moduły, zmienne typowanie, argumenty i funkcje czy też opcjonalne parametry funkcji. \textit{TypeScript}, poprzez zastosowanie m.in. silnego typowania znacząco przyspiesza pracę programistów i eliminuje pojawiające się błędy, przez co z roku na rok nabywa on coraz większą rzeszę zwolenników. \cite{TypeScript}

\subsection{Lodash}

\subsection{HTML5}
Język HTML5 (\textit{HyperText Markup Language}) jest najnowszą wersją popularnego standardu opisującego język HTML. Wprowadza on nowe elementy, atrybuty, zachowania i znacznie większy zestaw nowoczesnych technologii, umożliwiających budowanie bardziej zróżnicowanych i zaawansowanych technologicznie stron internetowych. Według powszechnej opinii jest to jedna z najlepszych rzeczy jakie spotkały dzisiejszy Internet. Jedną z najbardziej spektakularnych zmian w najnowszej wersji języka HTML jest natywna obsługa audio (znacznik \texttt{<audio>}) oraz wideo (znacznik \texttt{<video>}) bez użycia modułu \textit{Flash}. Ponadto, oferuje on dużo prostsze tworzenie i wyświetlanie grafiki przy użyciu znaczników \texttt{<canvas>}. Kolejnym udogodnieniem jest możliwość otwierania stron internetowych bez aktywnego połączenia z Internetem. Ponadto, została dodana opcja edycji dokumentów online (stosowana m.in. przez firmę \textit{Google}) czy też mechanizm fizycznego przenoszenia plików do okna przeglądarki internetowej zwany także jako \textit{Drag and Drop}. HTML5 wprowadza znaczniki (\texttt{<header>}, \texttt{<section>}, \texttt{<footer>}) ułatwiające ustrukturyzowanie elementów strony internetowej. Współpracuje on z wieloma rodzajami przeglądarek, jednakże nie wszystkie z nich mogą zapewnić stu procentową kompatybilność z tym językiem. Poprzez współpracę i interakcję z wieloma powszechnie znanymi technologiami tworzenia stron internetowych HTML5 często jest zwany jako "HTML5 i inne powiązane standardy". Sama składania języka nie jest zbyt skomplikowana. Kod można tworzyć w najprostszym edytorze tekstu. Warto także wspomnieć o tym, że HTML5 obsługuje tzw. \textit{Web-Workers}, które umożliwiają wielowątkową obsługę przepływu danych. \cite{Html}

\subsection{CSS3}
CSS (\textit{Cascading Style Sheets}) jest to framework służący do zwiększenia funkcjonalności, wydajności i poprawy wizualnej witryn WWW. Nie jest to \textit{stricte} język programistyczny - wymaga od swoich użytkowników myślenia typowo abstrakcyjnego oraz kreatywności. Pozwala on na tworzenie interfejsu GUI (\textit{Graphical User Interface}) użytkownika przy zachowaniu wysokiej szybkości działania aplikacji i jednoczesnej lekkości pisanego kodu. CSS używany jest w połączeniu z językiem HTML, który nadaje stronie określoną strukturę. Kaskadowe arkusze stylów w wersji trzeciej wprowadzają szereg udogodnień do nadawania wizualnej szaty stronom internetowym. Są nimi m.in. pozycjonowanie zawartości strony, ramki dla elementów, nowe właściwości dla tła, gradienty, cienie, transformacje 2D oraz 3D, lekkie i wydajne animacje, wielokolumnowy \textit{layout}, rewolucyjny \textit{Flex Box} czy przydatne przy tworzeniu responsywnych stron internetowych \textit{Media Queries}, pozwalające dopasowywać zawartości witryn do różnych rozdzielczości ekranów. Jest to najbardziej powszechny i znany na świecie język tworzenia UI (\textit{User Interface}) stron WWW. Bariera wejście w świat CSS jest zaskakująco niska. Należy jednak pamiętać o tym, że złożoność jego kodu rośnie wprost proporcjonalnie do wielkości projektu, w którym jest stosowany. Istnieje wiele preprocesorów języka CSS, które w znacznym stopniu ułatwiają jego stosowanie poprzez narzucanie określonej struktury i wprowadzanie możliwości tworzenia zmiennych, domieszek, operacji czy funkcji. Dwa z nich, najczęściej używane przez programistów to LESS (\textit{Leaner CSS}) oraz SASS (\textit{Syntatically Awesome Style Sheets}). \cite{Css}

\subsection{Bootstrap 4}
\textit{Bootstrap} to otwarte narzędzie (framework) do tworzenia stron WWW, przy użyciu języka HTML, CSS i \textit{JavaScript}, stworzone przez programistów popularnego serwisu \textit{Twitter}. Używany jest do tworzenia responsywnych witryn internetowych w podejściu \textit{Mobile First} kładącym duży nacisk na dostosowywanie widoku strony do urządzeń mobilnych. Umożliwia projektowanie stron WWW opartych na wierszowo-kolumnowym systemie siatek (twz. \textit{Grid}) mającym na celu usystematyzowanie położenia elementów witryny. Jest całkowicie darmowy do pobrania i codziennego użytku. W wersji czwartej frameworka dołożone zostały nowe komponenty. Przyspieszono również znacząco arkusze stylów i położono w niej jeszcze większy nacisk na responsywność. \textit{Bootstrap 4} jest wspierany przez wszystkie najnowsze wersje przeglądarek internetowych. W porównaniu z poprzednią, trzecią wersją \textit{Bootstrap 4} wprowadza m.in. prostą możliwość zmiany wielkości nagłówka H1, tzw. \textit{Flexbox} czyli nowy sposób układania elementów na stronie, nowy wygląd przycisków, karty, łatwiejszy sposób zmiany tła paska nawigacji czy łatwiejsze manipulowanie marginesami elementów. Jest to jednak wierzchołek góry lodowej jeżeli chodzi o zaistniałe zmiany. Możliwość edytowania i bezpośredniego wpływu na każdy z elementów biblioteki czyni ją niezastąpionym narzędziem w jeszcze szybszym i bardziej kreatywnym tworzeniu stron WWW. \cite{Bootstrap}

\subsection{Node.js}
\textit{Node.js} jest w pełni darmowym, otwartym, opartym na języku \textit{JavaScript} uruchomieniowym środowiskiem dla aplikacji serwerowych. Działa na wielu platformach, takich jak \textit{Windows}, \textit{Linux} czy \textit{MacOS}. Bazuje na silniku \textit{Google V8 JavaScript engine} napisanym w języku C++, pracującym w połączeniu z mechanizmem jednowątkowej pętli zdarzeń i niskopoziomowym interfejsem wejść/wyjść API (\textit{Application Programming Interface}). Silnik V8 stworzony przez firmę \textit{Google} pozwala na kompilowanie kodu \textit{JavaScript} do niskopoziomowego języka maszynowego. \textit{Node.js} wprowadza możliwość obsługi dużej ilości zapytań co znacząco wpływa na skalowalność aplikacji, w których jest on wykorzystywany. Środowisko oparte jest na opracowanym przez firmę \textit{Ecma International} standardzie \textit{ECMAScript}, na którym bazuje język \textit{JavaScript}. \textit{Node.js} jest mocno ustandaryzowany. Jego popularność wynika z tego, że język, na którym jest oparty jest w dzisiejszych czasach powszechnie stosowany. Dodatkowo, współpracuje z wieloma dobrze dopracowanymi frameworkami, takimi jak \textit{Express.js} co czyni go jeszcze bardziej niezastąpionym. Umożliwia on sprawne zarządzanie instalowanymi paczkami wraz z ich zależnościami dzięki dostarczanemu, dedykowanemu oprogramowaniu NPM (\textit{Node Package Manager}). Dzięki takiemu rozwiązaniu można w bardzo szybki i prosty sposób znaleźć odpowiadającą nam bibliotekę, przeszukując powszechnie dostępne, internetowe repozytorium NPM. Aplikacje \textit{Node.js} mogą być uruchamiane zarówno na urządzeniach IoT (\textit{Internet of Things}), jak i na powszechnie wykorzystywanych urządzeniach mobilnych, pracująch na procesorach architektury ARM (\textit{Advanced RICS Machine}). W dzisiejszych czasach środowisko to jest używane przez wielu gigantów biznesu internetowego, takich jak \textit{Google}, \textit{Microsoft}, \textit{Netflix}, \textit{PayPal} czy \textit{LinkedIn}. \cite{Node}

\subsection{Express.js}
\textit{Express.js} jest darmowym, otwartym frameworkiem przeznaczonym do współpracy ze środowiskiem serwerowym \textit{Node.js}. Jest zaprojektowany do tworzenia aplikacji internetowych oraz przeznaczonych dla nich API (\textit{Application Programming Interface}). Jest standardem typu \textit{de facto} zastosowań serwerowych bazujących na frameworku \textit{Node.js}. \textit{Express.js} pozwala definiować tabele routingu w celu wykonywania różnego typu akcji w oparciu o metodę HTTP. Umożliwia ponadto dynamiczne renderowanie stron napisanych w języku HTML w oparciu o przekazywanie argumentów do szablonów. \cite{Express}

\subsection{Karma}
Narzędzie zbudowane w oparciu o serwer \textit{NodeJS} oraz technologię \textit{Socket.io}, służące do automatycznego uruchamiania testów tworzonych w języku \textit{JavaScript} w emulowanym środowisku przeglądarek internetowych. Za pomocą \textit{Karma} możemy uruchamiać testy na różnych środowiskach programistycznych - deweloperskim, produkcyjnym czy testowym. Przy użyciu narzędzia \textit{Istanbul} możliwy jest dostęp do informacji na temat pokrycia implementowanego kodu testami. \textit{Karma} jest pełnoprawnym środowiskiem testowym ułatwiającym szybkie i bezproblemowe testowanie kodu \textit{JavaScript}.

\subsection{Jasmine}
Framework typu \textit{behavior-driven development framework} dający programistom wiele przydatnych funkcji potrzebnych do testowania oprogramowania. Jest zintegrowany ze środowiskiem \textit{Karma} i pozwala na pisanie testów oprogramowania w sposób opisowy. Nie jest on zależny od środowiska \textit{JavaScript} i nie wymaga drzewa DOM (\textit{Document Object Model}). Jasmine pozwala nie tylko na pisanie testów jednostkowych, ale także testów typu \textit{e2e} (\textit{end-to-end}). 

\subsection{Apache Cordova}
\textit{Cordova} to w skrócie API (\textit{Application Programming Interface}) umożliwiające stworzenie natywnej aplikacji używając wyłącznie HTML, CSS oraz kodu \textit{JavaScript} przy jednoczesnym dostępie do komponentów urządzeń mobilnych, takich jak aparat, usługi geolokalizacji czy nawet książki kontaktów. Aplikacje stworzone w ten sposób mogą znaleźć się na urządzeniach mobilnych producentów najbardziej wiodących firm - \textit{iOS}, \textit{Android}, \textit{Windows Phone} czy \textit{BlackBerry}.

\subsection{LaTeX}
Oprogramowanie służące do tworzenia przejrzystych dokumentów tekstowych takich jak na przykład książki czy artykuły. Docelowym plikiem wyjściowym jest najczęściej plik w formacie PDF (\textit{Portable Document Format}). Cechą charakterystyczną jest tutaj fakt, że \textit{LaTeX} ma swój własny język programowania, za pomocą którego tworzone są dokumenty. Do wygenerowania dokumentów przydatne są narzędzia przetwarzające pliki źródłowe i generujące dokumenty wyjściowe. \textit{LaTeX} oferuje dostęp do szeregu pakietów umożliwiających znacznie prostsze i szybsze implementowanie bardziej złożonych elementów plików wyjściowych. Filozofia \textit{LaTeXa} zakłada, aby skupiać się nie na tym jak dokument ma wyglądać, a co ma zawierać. Do użytkownika należy tylko wprowadzenie struktury i zawartości dokumentu. 

\subsection{PBKDF2}
PBKDF2 (\textit{Password-Based Key Derivation Function 2}) jest popularnym algorytmem podobnym do algorytmu \textit{BCrypt}, zapewniającym porównywalny stopień bezpieczeństwa. PBKDF2 jest bezpieczną funkcją skrótu stosowaną w celach zabezpieczania bezprzewodowych sieci WiFi (WPA, WPA2). Algorytm ten w celu wygenerowania klucza pochodnego wykorzystuje pseudolosową funkcję, taką jak HMAC (\textit{Hash Message Authentication Code}) powtarzając operacje hashowania przez określoną, dużą liczbę iteracji. Algorytm PBKDF2 jest trudny do złamania za pomocą CPU (\textit{Central Processing Unit}), ale nie wymaga zbyt dużych zasobów pamięciowych i łatwo jest go zrównoleglić za pomocą GPU (\textit{Graphics Processing Unit}). Mimo to, że jest on wciąż stosowany to nie zaleca się używania go przy implementacji nowych projektów. 

\subsection{Argon2}
Algorytm \textit{Argon2} został zwycięzcą \textit{Password Hashing Competition} i jako następca \textit{BCrypt} oraz \textit{Scrypt} jest obecnie zalecany do zabezpieczania haseł. \textit{Argon2} zawiera szereg zabezpieczeń przeciwko atakom typu \textit{Brute Force}. W przeciwieństwie do \textit{PBKDF2} algorytm ten wprowadza silną odporność nie tylko na ataki przy użyciu CPU, ale także GPU (wykorzystuje się konkretną odmianę \textit{Argon2d}). Samo użycie algorytmu jest niezwykle proste. Wiele języków programowania oferuje dedykowane bilbioteki, które znacząco ułatwiają używanie \textit{Argon2} w implementacjach. Algorytm występuje w dwóch bazowych wersjach - \textit{Argin2i} oraz \textit{Argon2d} chroniącym przed atakami typu \textit{GPU cracking}. Istnieje również hybrydowe połączenie dwóch rodzajów funkcji zwane jako \textit{Argon2id}. \textit{Argon2} współpracuje z 64 - bitowymi architekturami procesorów i powinien być kompilowany na systemach takich jak \textit{Linux}, \textit{OS X} czy \textit{Windows}. 

\section{Narzędzia}
\subsection{Visual Studio Code 1.28}
\textit{Visual Studio Code} jest wieloplatformowym, prostym w obsłudze IDE (\textit{Integrated Development Environment}) stworzonym przez firmę \textit{Microsoft}. Łączy on w sobie prostotę edytora kodu źródłowego z potężnym środowiskiem deweloperskim oferującym mechanizm \textit{IntelliSense} czy mechanizm debugowania kodu. \textit{Visual Studio Code} oferuje szereg skrótów klawiszowych i snippetów ułatwiających szybsze i bardziej intuicyjne implementowanie oprogramowania. Autorzy udostępnili mnóstwo udogodnień wspierających pracę w zespole, takich jak integracja z systemem kontroli wersji \textit{Git} czy też rozproszone współdzielenie kodu. Oprogramowanie połączone jest bezpośrednio z uaktualnianym na bieżąco repozytorium paczek i pakietów ułatwiających programowanie w danym języku i technologii. Dodatkowo, \textit{VSCode} oferuje przyjazne GUI (\textit{Graphical User Interface}), z przejrzystym eksploratorem drzewa plików i mapą zawartości pliku, ułatwiającą szybsze wykrycie błędów i ostrzeżeń w kodzie \cite{Vsc}.

\subsection{github.com}
Serwis internetowy stworzony dla projektów programistycznych, który wykorzystuje system kontroli wersji \textit{Git}. Jego implementacja ma podłoże w języku \textit{Erlang} z wykorzystaniem frameworka \textit{Ruby on Rails}. Github umożliwia darmowy hosting plików oraz płatne, prywatne repozytoria. Platforma oferuje szereg statystyk powiązanych z implementowanym kodem źródłowym, mechanizm typu \textit{bugtracker}, możlwiość pobierania repozytoriów, rozgałęziania (dzielenia) pracy pomiędzy członków zespołu programistycznego czy późniejszego ich łączenia. Dodatkowo serwis ten oferuje usługę zwaną \textit{Github Pages} służącą do szybkiego tworzenia stron internetowych kompilowanych na podstawie kodu zawartego w repozytorium. Dzięki \textit{github.com} mamy możliwość bezpośredniej kontroli nad tworzonym kodem oprogramowania i dostęp do historii pracy co znacząco ułatwia zarządzanie projektem programistycznym. 

\subsection{Auth0}
Serwis \textit{Auth0} umożliwia połączenie z dowolną aplikacją w celu zaoferowania usług uwierzytelniania użytkowników. Oferuje on metody logowania i rejestracji w serwisie za pomocą tradycyjnego loginu (adresu e-mail) i hasła, bądź mediów społecznościowych takich jak \textit{Google}, \textit{Facebook} czy \textit{Twitter}. Domyślny protokół używany do integracji serwisu \textit{Auth0} z aplikacją użytkownika i~ pózniejszego uwierzytelniania to OIDC (\textit{OpenID Connect}). Używa on prostych \textit{tokenów} identyfikacyjnych w formacie JSON (\textit{JavaScript Object Notation}). Wymiana danych odbywa się przy użyciu JWT (\textit{JSON Web Token}), który zawiera wszystkie dane identyfikacyjne użytkownika \cite{Auth}.

\subsection{TexStudio}
\textit{TexStudio} jest zintegrowanym środowiskiem służącym do tworzenia dokumentów w języku \textit{LaTeX}. Program posiada szereg funkcji mających na celu ułatwienie tworzenia tekstów, a w ich skład wchodzą między innymi podświetlanie składni, zintegrowana przeglądarka, system sprawdzania referencji czy zintegrowane, zewnętrzne repozytorium pakietów i rozszerzeń języka \textit{LaTeX}.